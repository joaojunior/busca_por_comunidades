\documentclass{proc}

\usepackage{edbkps}
\usepackage{amsmath}
\usepackage{amssymb}
\usepackage{setspace}
\usepackage[utf8]{inputenc}
\usepackage{setspace}
\usepackage{subfig}
%%\usepackage[lined,boxed,commentsnumbered, linesnumbered]{algorithm2e}
\usepackage[lined,boxed,commentsnumbered, linesnumbered, portuguese]{algorithm2e}



\setcounter{secnumdepth}{2}
\setcounter{tocdepth}{0}
\let\footnote\savefootnote
\let\footnotetext\savefootnotetext
\let\footnoterule\savefootnoterule
\normallatexbib
\makeatletter
\renewcommand{\@ptsize}{0}
\makeatother

%%%%%
%% If you use a font encoding package, please enter it here, i.e.,
\usepackage{t1enc}


%%%%%
% Style for inserting .eps files and rotating illustrations or tables

% possible options for graphicx:
% [dvips], [xdvi], [dvipdf], [dvipsone], [dviwindo], [emtex], [dviwin],
% [pctexps],  [pctexwin],  [pctexhp],  [pctex32], [truetex], [tcidvi],
% [oztex], [textures]

\usepackage[dvips]{graphicx}
\usepackage[x11names, rgb]{xcolor}
\usepackage{tikz}
\usetikzlibrary{snakes,arrows,shapes}
\usepackage{amsmath}

%%%%%%%%%%%%%%%%%%%%%%%%%%%%%%%%%%%%%%%%%%%%%%%%%%%%%%%%%%%%%%%%%%%%%%%%%

\begin{document}


\articletitle[Comunidades em Grafos]
{Comunidades em Grafos}

\author{João C. Abreu\altaffilmark{1}}

\altaffiltext{1}{Universidade Federal de Minas Gerais, DCC, \\
Avenida Antônio Carlos 6627, Belo Horizonte, Brazil,\\
\email{joao.junior@dcc.ufmg.br}}

\anxx{João C. Abreu Jr.,Universidade Federal de Minas Gerais\,
Avenida Antônio Carlos 6627\, Belo Horizonte\, Brasil\,
\emailfont{joao.junior@dcc.ufmg.br}}

\begin{abstract}
Esse trabalho apresenta o problema de calcular comunidades e cinco algoritmos exatos para a resolução desse problema.

\end{abstract}

\begin{keywords}
\inx{Comunidades em Grafo}, \inx{Centralidade}, \inx{betweenness}
\end{keywords}

\doublespacing

\section{Introdução}
\label{sec:intro}

Seja um grafo $G = (V,E)$, onde $V$ é o conjunto de vértices e $E$ é o conjunto de arestas, com a cardinalidade $|V| = n$ e 
$|E| = m$. O peso de cada aresta $(i, j) \in E$ é dado pela função $w(i, j) = 1, \forall (i, j) \in E$. A
centralidade (do inglês \emph{betweenness}) de uma aresta $(i, j) \in E$ é definida como a quantidade de caminhos mínimos que utilizam
a aresta $(i, j) \in E$. O problema de encontrar $k$ comunidades em um grafo $G$ pode ser visto como um problema de se 
retirar as arestas com maior centralidade no grafo até possuir $k$ componentes conexas nesse grafo.
O Objetivo desse trabalho é comparar cinco algoritmos exatos para o problema de cálculos de comunidade. 
A seção \ref{sec:estrutura_dados} apresenta as estruturas de dados utilizadas nesse trabalho, a seção \ref{sec:algoritmos}
apresenta os cinco algoritmos exatos para o problema de cálculos de comunidades, a seção \ref{sec:experimentos}
apresenta os experimentos computacionais desse trabalho.


Para se encontrar $k$ comunidades em um grafo, com $k < m$, calcula-se a centralidade de cada 
aresta $(i, j) \in E$ e retira-se do grafo a aresta $(i, j) \in E$ que possuir a maior centralidade. Após isso a quantidade de componentes conexas do grafo é verificada e se for igual a $k$ o problema está resolvida, caso contrário o cálculo de centralidades
deve ser novamente efetuado e a aresta $(i, j) \in E$ com maior centralidade é removida do grafo.


Todo código fonte produzido por esse trabalho pode ser obtido no endereço eletrônico: $https://github.com/joaojunior/feixo1\_2scp$

\section{Estrutura de Dados}\label{sec:estrutura_dados} 
Essa seção apresenta as estruturas de dados que serão utilizadas nos algoritmos apresentados
na seção \ref{sec:algoritmos}. A seção \ref{sec:grafo} mostra a estrutura de dados que representa
um grafo $G =(V, E)$ e a seção \ref{sec:queue_min} apresenta a estrutura de dados fila de prioridades
mínimas.

\subsection{Grafo}\label{sec:grafo} 

Um grafo $G = (V, E)$ vai ser representado por uma matriz $n$ x $n$, onde $n = |V|$. Quando existe uma 
aresta $(i, j) \in E$ o valor armazenado na posição $i, j$ da matriz será $1$ e quando a posição 
$i, j$ da matriz não corresponder a uma aresta será armazenado o valor do maior inteiro possível.
Assim a ordem de complexidade da quantidade de memória utilizada para representar um grafo vai ser igual a $\theta(n ^ 2)$ e 
a ordem de complexidade para verificar se uma aresta existe no grafo será $O(1)$. A estrutura de dados grafo está
definida no arquivo graph.h e sua implementação pode ser obtida no arquivo graph.c.

\subsection{Fila de prioridade Mínima}\label{sec:queue_min}
A fila de prioridade mínima \cite{cormen3ndqueuemin} vai ser representada por um vetor de $n$ elementos. Cada elemento armazena dois inteiros, um
representa o peso do elemento e o outro a posição na fila de prioridade mínima. Um elemento pode possuir no máximo dois filhos, onde
o peso do elemento dos filhos é igual ou menor do que o peso do elemento pai. Assim a ordem de complexidade da quantidade
de memória para armazenar uma fila de prioridade mínima é $\theta(n)$, a ordem de complexidade para se retirar o elemento de menor
peso da fila e para se diminuir o valor de um peso de um elemento existente na fila é $O(log_2 n)$. A estrutura de dados fila 
de prioridade mínima está definida no arquivo min\_priority\_queue.h e sua implementação pode ser 
obtida no arquivo min\_priority\_queue.c.
\section{Algoritmos}\label{sec:algoritmos} 
Essa seção apresenta um algoritmo genérico para o cálculo de comunidades e análisa a ordem de complexidade de cinco algoritmos baseados nesse algoritmo genérico.
O Algoritmo \ref{algoritmo_generico} apresenta um algoritmo genérico para o cálculo de comunidades em um grafo $G = (V, E)$.
Na linha 2 é calculado a centralidade de todas as arestas, o laço das linhas 3 a 6 é repetido até que o número $k$ de comunidades
é obtido. A linha 4 retira do grafo $G = (V, E)$ a aresta $(i, j) \in E$ que possui a maior centralidade. A linha 5 recalcula a centralidade de todas as arestas.
As seções \ref{sec:faster}, \ref{sec:floydwarshall}, \ref{sec:johnson_queue}, \ref{sec:johnson_vector} e \ref{sec:bfs} 
analisam a ordem de complexidade dos algoritmos que se diferem do algoritmo genérico apenas pela cálculo de centralidades. As seções 
\ref{sec:faster}, \ref{sec:floydwarshall}, \ref{sec:johnson_queue}, \ref{sec:johnson_vector} e \ref{sec:bfs} 
apresentam as análises dos algoritmos que calculam a centralidade utilizando, respectivamente, os algoritmos: Faster-All-Pairs-Shortest-Path\cite{cormen3ndFaster}, Floyd-Warshall\cite{cormen3ndFloydwarshall}, Johnson com fila de prioridade mínima\cite{cormen3ndjohnson}, Johnson com array\cite{cormen3ndjohnson} e Breadth First Search\cite{cormen3ndbfs}.

\begin{algorithm}
\Entrada{Grafo $G = (V, E)$ e número $k$ de comunidades}
\Saida{Identificador da comunidade e label do nó em cada comunidade}
\Inicio{ \nonumber
    Calcular centralidades de todas as arestas $(i, j) \in E$\\
    \While{o número de comunidades < $k$}{
        Remover a aresta que possuir a maior centralidade \\
        Recalcular a centralidade de todas as arestas \\
    }
}            
\caption{Algoritmo genérico para o cálculo de comunidades em um grafo}
\label{algoritmo_generico}
\end{algorithm}

\subsection{Faster-All-Pairs-Shortest-Path}\label{sec:faster}
O cálculo de centralidades é feito utilizando o algoritmo Faster-All-Pairs-Shortest-Path. O Algoritmo Faster-All-Pairs-Shortest-Path
possui ordem de complexidade igual a $\theta(n^3log_2n)$. No melhor caso apenas $k$ arestas serão retiradas do grafo e no pior
caso $m$ arestas serão retiradas do grafo. Assim o algoritmo de cálculo de comunidades utilizando o algoritmo Faster-All-Pairs-Shortest-Path terá ordem de complexidade igual a $\theta(kn^3log_2n)$ no melhor caso e $\theta(mn^3log_2n)$ no pior caso. A quantidade
de memória utilizada por esse algoritmo será da ordem de $\theta(n^2)$, que é a ordem de utilização de memória dado pela estrutura
de dados grafo.

\subsection{Floyd-Warshall}\label{sec:floydwarshall}
O cálculo de centralidades é feito utilizando o algoritmo de Floydwarshall. O Algoritmo Floydwarshall
possui ordem de complexidade igual a $\theta(n^3)$. No melhor caso apenas $k$ arestas serão retiradas do grafo e no pior
caso $m$ arestas serão retiradas do grafo. Assim o algoritmo de cálculo de comunidades utilizando o algoritmo Floydwarshall terá ordem de complexidade igual a $\theta(kn^3)$ no melhor caso e $\theta(mn^3)$ no pior caso. A quantidade
de memória utilizada por esse algoritmo será da ordem de $\theta(n^2)$, que é a ordem de utilização de memória dado pela estrutura
de dados grafo.

\subsection{Johnson com Fila de Prioridades Mínimas}\label{sec:johnson_queue}
O cálculo de centralidades é feito utilizando o algoritmo de Johnson que utiliza o algoritmo de Dijkstra\cite{cormen3nddijkstra} com fila
de prioridade mínima. Nesse caso o Algoritmo de Johnson que utiliza o algoritmo de Dijkstra com fila de prioridade mínima
possui ordem de complexidade igual a $O(nmlog_2n)$. No melhor caso apenas $k$ arestas serão retiradas do grafo e no pior
caso $m$ arestas serão retiradas do grafo. Assim o algoritmo de cálculo de comunidades utilizando o algoritmo de Johnson terá ordem de complexidade igual a $O(knmlog_2n)$ no melhor caso e $O(nm^2log_2n)$ no pior caso. A quantidade
de memória utilizada por esse algoritmo será da ordem de $\theta(n^2)$, que é a ordem de utilização de memória dado pela estrutura
de dados grafo.

\subsection{Johnson com Array}\label{sec:johnson_vector}
O cálculo de centralidades é feito utilizando o algoritmo de Johnson que utiliza o algoritmo de Dijkstra\cite{cormen3nddijkstra} com um array. Nesse caso o Algoritmo de Johnson possui ordem de complexidade igual a $\O(n^3 + nm)$. No melhor caso apenas $k$ arestas serão retiradas do grafo e no pior caso $m$ arestas serão retiradas do grafo. Assim o algoritmo de cálculo de comunidades utilizando o algoritmo de Johnson terá ordem de complexidade igual a $O(k(n^3 + nm))$ no melhor caso e $O(mn^3 + nm^2)$ no pior caso. A quantidade
de memória utilizada por esse algoritmo será da ordem de $\theta(n^2)$, que é a ordem de utilização de memória dado pela estrutura
de dados grafo.

\subsection{Breadth First Search}\label{sec:bfs}
O cálculo de centralidades é feito utilizando o algoritmo Breadth First Search (BFS). O Algoritmo BFS
possui ordem de complexidade igual a $O(n + m)$. No cálculo de centralidade utilizando o algoritmo BFS, 
para cada vértice $v \in V$ é executado o algoritmo BFS. No melhor caso apenas $k$ arestas serão retiradas do grafo e no pior
caso $m$ arestas serão retiradas do grafo. Assim o algoritmo de cálculo de comunidades utilizando o algoritmo BFS terá ordem de complexidade igual a $O(k(n^2 + nm))$ no melhor caso e $O(mn^2 + nm^2))$ no pior caso. A quantidade
de memória utilizada por esse algoritmo será da ordem de $\theta(n^2)$, que é a ordem de utilização de memória dado pela estrutura
de dados grafo.
\section{Experimentos Computacionais}\label{sec:experimentos} 
Os experimentos computacionais foram executados em uma máquina Intel Dual-Core de 2.81 GHz de clock e
2GB de memória RAM, rodando o sistema operacional Linux. Os algoritmos das seções \ref{sec:faster}, \ref{sec:floydwarshall}, \ref{sec:johnson_queue}, \ref{sec:johnson_vector} e \ref{sec:bfs} foram implementados na 
linguagem de programação c e compilados com o compilador gcc na versão 4.8.2. A tabela \ref{table:algoritmos} apresenta esses cinco algoritmos. Na coluna 1 está o nome que o algoritmo vai assumir aqui, a coluna 2 é a referência da seção que o algoritmo foi apresentado,
a coluna 3 mostra o nome do arquivo de especificações do algoritmo e a coluna 4 apresenta o nome do arquivo com o código fonte do 
algoritmo.

\begin{table}[htbp]
\begin{center}
  \begin{tabular}{|c|c|c|c|}
    \hline
      Algoritmo                & Seção                    & Arquivo .h              & Arquivo .c            \\ \hline
      kc\_faster               & \ref{sec:faster}         & repeated\_squaring.h    & repeated\_squaring.c  \\ \hline
      kc\_floydwarshall        & \ref{sec:floydwarshall}  & floyd\_warshall.h       & floyd\_warshall.c     \\ \hline
      kc\_johnson\_min\_queue    & \ref{sec:johnson_queue}  & johnson.h               & johnson.c           \\ \hline
      kc\_johnson\_array        & \ref{sec:johnson_vector} & johnson.h               & johnson.c            \\ \hline
      kc\_nbfs                 & \ref{sec:bfs}            & bfs.h                   & bfs.c                 \\ \hline
  \end{tabular}
\caption{Algoritmos exatos para o problema de encontradar k comunidades em um grafo não direcionado e não ponderado}
\label{table:algoritmos}
\end{center}
\end{table}

Foram utilizados cinco conjuntos de instâncias de testes nos experimentos computacionais, três dessas instâncias foram retiradas de \cite{files_test} e as outras duas foram geradas por esse trabalho. As instâncias de teste retiradas de \cite{files_test}
foram: karate, lesmis e adjnoun. A instância karate representa uma rede social de 34 membros de um clube de karate, a instância
lesmis representa o número de ocorrências de determinados caracteres em um livro e a instância adjnoun representa
o número de ocorrências de palavras em um livro. As instâncias geradas vão ser chamadas de path e completo. Essas instâncias
foram geradas pois vão explorar o melhor e o pior caso dos algoritmos apresentados. Uma instância
path é um grafo onde o nó $i \in V$ está ligado ao nó $i+1 \in V$ formando apenas um caminho simples. Foram gerados três 
instâncias path com 50, 100 e 200 nós.  Uma instância completo é um grafo completo, ou seja, existe uma aresta
entre cada par de nós do grafo. Foram geradas três instâncias completo com 50, 100 e 200 nós.
A tabela \ref{table:instancias} apresenta todas as instâncias de teste. Na coluna 1 dessa tabela está o nome da instância,
a coluna 2 apresenta o número de nós da instância e a coluna 3 mostra o número de aresta de cada instância.

\begin{table}[htbp]
\begin{center}
  \begin{tabular}{|c|c|c|}
    \hline
      Instância                & \# Nós                   & \# Arestas \\ \hline
      karate                   & 34                       & 78 \\ \hline
      lesmis                   & 77                       & 254 \\ \hline
      adjnoun                  & 112                      & 425 \\ \hline
      path\_50                 & 50                       & 49 \\ \hline
      path\_100                & 100                      & 99 \\ \hline
      path\_200                & 200                      & 199 \\ \hline
      completo\_50             & 50                       & 2450 \\ \hline
      completo\_100            & 100                      & 9900 \\ \hline
      completo\_200            & 200                      & 39800 \\ \hline
  \end{tabular}
\caption{Características das instâncias de testes}
\label{table:instancias}
\end{center}
\end{table}

No primeiro experimento foi comparado a performance dos algoritmos das seções \ref{sec:faster}, \ref{sec:floydwarshall}, \ref{sec:johnson_queue}, \ref{sec:johnson_vector} e \ref{sec:bfs} nas instâncias retiradas de \cite{files_test}.
A figura \ref{karate} mostra os resultados obtidos com a execução dos algoritmos para a instância karate. Essa figura mostra
que o algoritmo mais rápido foi o de kc\_nbfs seguido do kc\_floydwarshall, essa figura também mostra que a medida
que o número de comunidades aumenta o algoritmo kc\_faster passa a ser o mais lento. O algoritmo kc\_johnson\_min\_queue
passa a ser mais rápido que o algoritmo kc\_johnson\_array com o aumento do número de comunidades.

\begin{figure}
\centering
\includegraphics[width=6in]{karate.png}
\caption{Comparação de tempo em segundos pelo número de comunidades:2, 4, 8, 16 e 32 para os cinco algoritmos apresentados nas seções \ref{sec:faster}, \ref{sec:floydwarshall}, \ref{sec:johnson_queue}, \ref{sec:johnson_vector} e \ref{sec:bfs} e excutado na instância de teste karate}
\label{karate}
\end{figure}

A figura \ref{lesmis} mostra os resultados obtidos com a execução dos algoritmos para a instância lesmis. Essa figura mostra
que o algoritmo mais rápido foi o de kc\_nbfs seguido do kc\_floydwarshall, essa figura também mostra que a medida
que o número de comunidades aumenta o algoritmo kc\_faster passa a ser o mais lento. O algoritmo kc\_johnson\_min\_queue
passa a ser mais rápido que o algoritmo kc\_johnson\_array com o aumento do número de comunidades.

\begin{figure}
\centering
\includegraphics[width=6in]{lesmis.png}
\caption{Comparação de tempo em segundos pelo número de comunidades:2, 4, 8, 16, 32 e 64 para os cinco algoritmos apresentados nas seções \ref{sec:faster}, \ref{sec:floydwarshall}, \ref{sec:johnson_queue}, \ref{sec:johnson_vector} e \ref{sec:bfs} e excutado na instância de teste lesmis}
\label{lesmis}
\end{figure}

A figura \ref{adjnoun} mostra os resultados obtidos com a execução dos algoritmos para a instância adjnoun. Essa figura mostra
que o algoritmo mais rápido foi o de kc\_nbfs seguido do kc\_floydwarshall, essa figura também mostra que a medida
que o número de comunidades aumenta o algoritmo kc\_faster passa a ser o mais lento. O algoritmo kc\_johnson\_min\_queue
passa a ser mais rápido que o algoritmo kc\_johnson\_array com o aumento do número de comunidades.

\begin{figure}
\centering
\includegraphics[width=6in]{adjnoun.png}
\caption{Comparação de tempo em segundos pelo número de comunidades:2, 4, 8, 16, 32 e 64 para os cinco algoritmos apresentados nas seções \ref{sec:faster}, \ref{sec:floydwarshall}, \ref{sec:johnson_queue}, \ref{sec:johnson_vector} e \ref{sec:bfs} e excutado na instância de teste adjnoun}
\label{adjnoun}
\end{figure}

No segundo experimento foi comparado a performance dos algoritmos das seções \ref{sec:faster}, \ref{sec:floydwarshall}, \ref{sec:johnson_queue}, \ref{sec:johnson_vector} e \ref{sec:bfs} no melhor e pior caso. No melhor caso, os algoritmos foram
executados nas instâncias path50, path100 e path200 com o número de comunidades $k$ igual a 50, 100 e 200 respectivamente, assim
todas as $k - 1$ arestas do grafo precisaram ser retiradas. No pior caso, os algoritmos foram executados nas instâncias completo50, completo100 e completo200 com número de comunidades $k$ igual a 50, 100 e 200 respectivamente, obrigando a
retirada de todas as arestas do grafo. Para comparar os algoritmos, o tempo de execução de cada um foi dividido pelo tempo 
de execução do algoritmo kc\_nbfs que foi o algoritmo que obteve o menor tempo de resposta.
A figura \ref{path} apresenta os resultados obtidos para as instâncias path50, path100 e path200.  Nessa figura o eixo
vertical representa o tempo que o algoritmo gastou dividido pelo tempo gasto pelo algoritmo kc\_nbfs e o eixo horizontal
representa cada uma das instâncias path. Obervando essa figura vemos que o algoritmo kc\_faster é o algoritmo que apresenta os piores resultados e com o aumento do número de arestas no grafo path a razão do tempo gasto por esse algoritmo e o algoritmo kc\_nbfs aumenta muito. O algoritmo kc\_floydwarshall consumiu tempos similares aos tempos do algoritmo kc\_nbfs. O algoritmo
kc\_johnson\_min\_queue foi mais rápido do que o algoritmo kc\_johnson\_array.

\begin{figure}
\centering
\includegraphics[width=6in]{path.png}
\caption{Comparação da razão do tempo gasto dos algoritmos pelo tempo gasto pelo algoritmo mais rápido nas instâncias path50, path100 e path200}
\label{path}
\end{figure}

A figura \ref{completo} apresenta os resultados obtidos para as instâncias completo50, completo100 e completo200.  Nessa figura o eixo vertical representa o tempo que o algoritmo gastou dividido pelo tempo gasto pelo algoritmo kc\_nbfs e o eixo horizontal
representa cada uma das instâncias completo. Obervando essa figura vemos que o algoritmo kc\_faster é o algoritmo que apresenta os piores resultados e com o aumento do número de arestas no grafo completo a razão do tempo gasto por esse algoritmo e o algoritmo kc\_nbfs aumenta muito. O algoritmo kc\_floydwarshall consumiu tempos similares aos tempos do algoritmo kc\_nbfs. O algoritmo
kc\_johnson\_min\_queue foi mais rápido do que o algoritmo kc\_johnson\_array.

\begin{figure}
\centering
\includegraphics[width=6in]{completo.png}
\caption{Comparação da razão do tempo gasto dos algoritmos pelo tempo gasto pelo algoritmo mais rápido nas instâncias completo50,
completo100 e completo200}
\label{completo}
\end{figure}

Comparando as figuras \ref{path} e \ref{completo} a razão entre o tempo dos algoritmos e o tempo do algoritmo kc\_nbfs diminui, isso pode ser explicado porque com o aumento do número de arestas no grafo o algoritmo kc\_nbfs tende a ser muito pior.












\bibliographystyle{plain}
%\chapbblname{tp}  % The name of the .bbl file, what is normally the name of your .tex file.
\bibliography{Tp} %\chapbibliography{gow}

\end{document}





% LocalWords:  comeca ij kj ji lll hardcoded eq ranqueamento recuperável RRSP
% LocalWords:  Recuperável Rent Minmax Regret Single Pair MSP SPP Cplex
