\documentclass{proc}

\usepackage{edbkps}
\usepackage{amsmath}
\usepackage{amssymb}
\usepackage{setspace}
\usepackage[utf8]{inputenc}
\usepackage{setspace}
\usepackage{subfig}
%%\usepackage[lined,boxed,commentsnumbered, linesnumbered]{algorithm2e}
\usepackage[lined,boxed,commentsnumbered, linesnumbered, portuguese]{algorithm2e}



\setcounter{secnumdepth}{2}
\setcounter{tocdepth}{0}
\let\footnote\savefootnote
\let\footnotetext\savefootnotetext
\let\footnoterule\savefootnoterule
\normallatexbib
\makeatletter
\renewcommand{\@ptsize}{0}
\makeatother

%%%%%
%% If you use a font encoding package, please enter it here, i.e.,
\usepackage{t1enc}


%%%%%
% Style for inserting .eps files and rotating illustrations or tables

% possible options for graphicx:
% [dvips], [xdvi], [dvipdf], [dvipsone], [dviwindo], [emtex], [dviwin],
% [pctexps],  [pctexwin],  [pctexhp],  [pctex32], [truetex], [tcidvi],
% [oztex], [textures]

\usepackage[dvips]{graphicx}
\usepackage[x11names, rgb]{xcolor}
\usepackage{tikz}
\usetikzlibrary{snakes,arrows,shapes}
\usepackage{amsmath}

%%%%%%%%%%%%%%%%%%%%%%%%%%%%%%%%%%%%%%%%%%%%%%%%%%%%%%%%%%%%%%%%%%%%%%%%%

\begin{document}


\articletitle[Comunidades em Grafos]
{Comunidades em Grafos}

\author{João C. Abreu\altaffilmark{1}}

\altaffiltext{1}{Universidade Federal de Minas Gerais, DCC, \\
Avenida Antônio Carlos 6627, Belo Horizonte, Brazil,\\
\email{joao.junior@dcc.ufmg.br}}

\anxx{João C. Abreu Jr.,Universidade Federal de Minas Gerais\,
Avenida Antônio Carlos 6627\, Belo Horizonte\, Brasil\,
\emailfont{joao.junior@dcc.ufmg.br}}

\begin{abstract}
Esse trabalho apresenta o problema de calcular comunidades e cinco algoritmos exatos para a resolução desse problema.

\end{abstract}

\begin{keywords}
\inx{Comunidades em Grafo}, \inx{Centralidade}, \inx{betweenness}
\end{keywords}

\doublespacing

\section{Introdução}
\label{sec:intro}

Seja um grafo $G = (V,E)$, onde $V$ é o conjunto de vértices e $E$ é o conjunto de arestas, com a cardinalidade $|V| = n$ e 
$|E| = m$. O peso de cada aresta $(i, j) \in E$ é dado pela função $w(i, j) = 1, \forall (i, j) \in E$. A
centralidade (do inglês \emph{betweenness}) de uma aresta $(i, j) \in E$ é definida como a quantidade de caminhos mínimos que utilizam
a aresta $(i, j) \in E$. O problema de encontrar $k$ comunidades em um grafo $G$ pode ser visto como um problema de se 
retirar as arestas com maior centralidade no grafo até possuir $k$ componentes conexas nesse grafo.
O Objetivo desse trabalho é comparar cinco algoritmos exatos para o problema de cálculos de comunidade. 
A seção \ref{sec:estrutura_dados} apresenta as estruturas de dados utilizadas nesse trabalho, a seção \ref{sec:algoritmos}
apresenta os cinco algoritmos exatos para o problema de cálculos de comunidades, a seção \ref{sec:experimentos}
apresenta os experimentos computacionais desse trabalho.


Para se encontrar $k$ comunidades em um grafo, com $k < m$, calcula-se a centralidade de cada 
aresta $(i, j) \in E$ e retira-se do grafo a aresta $(i, j) \in E$ que possuir a maior centralidade. Após isso a quantidade de componentes conexas do grafo é verificada e se for igual a $k$ o problema está resolvida, caso contrário o cálculo de centralidades
deve ser novamente efetuado e a aresta $(i, j) \in E$ com maior centralidade é removida do grafo.


Todo código fonte produzido por esse trabalho pode ser obtido no endereço eletrônico: $https://github.com/joaojunior/feixo1\_2scp$

\section{Estrutura de Dados}\label{sec:estrutura_dados} 
Essa seção apresenta as estruturas de dados que serão utilizadas nos algoritmos apresentados
na seção \ref{sec:algoritmos}. A seção \ref{sec:grafo} mostra a estrutura de dados que representa
um grafo $G =(V, E)$ e a seção \ref{sec:queue_min} apresenta a estrutura de dados fila de prioridades
mínimas.

\subsection{Grafo}\label{sec:grafo} 

Um grafo $G = (V, E)$ vai ser representado por uma matriz $n$ x $n$, onde $n = |V|$. Quando existe uma 
aresta $(i, j) \in E$ o valor armazenado na posição $i, j$ da matriz será $1$ e quando a posição 
$i, j$ de uma matriz não corresponder a uma aresta será armazenado o valor do maior inteiro possível.
Assim a ordem de complexidade da quantidade de memória utilizada para representar um grafo vai ser igual a $\theta(n ^ 2)$ e 
a ordem de complexidade para verificar se uma aresta existe no grafo será $O(1)$. A estrutura de dados grafo está
definida no arquivo graph.h e sua implementação pode ser obtida no arquivo graph.c.

\subsection{Fila de prioridade Mínima}\label{sec:queue_min}
A fila de prioridade mínima vai ser representada por um vetor de $n$ elementos. Cada elemento armazena dois inteiros, um
representa o peso do elemento e o outro a posição na fila de prioridade mínima. Um elemento pode possuir no máximo dois filhos, onde
o peso do elemento dos filhos é igual ou menor do que o peso do elemento pai. Assim a ordem de complexidade da quantidade
de memória para armazenar uma fila de prioridade mínima é $\theta(n)$, a ordem de complexidade para se retirar o elemento de menor
peso da fila e para se diminuir o valor de um peso de um elemento existente na fila é $O(log_2 n)$.
\section{Algoritmos}\label{sec:algoritmos} 
Essa seção apresenta um algoritmo genérico para o cálculo de comunidades e análisa a ordem de complexidade de cinco algoritmos baseados nesse algoritmo genérico.
O Algoritmo \ref{algoritmo_generico} apresenta um algoritmo genérico para o cálculo de comunidades em um grafo $G = (V, E)$.
Na linha 2 é calculado a centralidade de todas as arestas, o laço das linhas 3 a 6 é repetido até que o número $k$ de comunidades
é obtido. A linha 4 retira do grafo $G = (V, E)$ a aresta $(i, j) \in E$ que possui a maior centralidade. A linha 5 recalcula os
caminhos mínimos entre todos os pares de vértices $i, j \in V$.
As seções \ref{sec:faster}, \ref{sec:floydwarshall}, \ref{sec:johnson_queue}, \ref{sec:johnson_vector} e \ref{sec:bfs} 
analisam a ordem de complexidade dos algoritmos que se diferem do algoritmo genérico apenas pela cálculo de centralidades. As seções 
\ref{sec:faster}, \ref{sec:floydwarshall}, \ref{sec:johnson_queue}, \ref{sec:johnson_vector} e \ref{sec:bfs} 
apresentam as análises dos algoritmos que calculam a centralidade utilizando, respectivamente, os algoritmos: Faster-All-Pairs-Shortest-Path, Floyd-Warshall, Johnson com fila de prioridades mínimas, Johnson com array e Breadth First Search.

\begin{algorithm}
\Entrada{Grafo $G = (V, E)$ e número $k$ de comunidades}
\Saida{Identificador da comunidade e label do nó em cada comunidade}
\Inicio{ \nonumber
    Calcular centralidades de todas as arestas $(i, j) \in E$\\
    \While{o número de comunidades < $k$}{
        Remover a aresta que possuir a maior centralidade \\
        Recalcular a centralidade de todas as arestas \\
    }
}            
\caption{Algoritmo genérico para o cálculo de comunidades em um grafo}
\label{algoritmo_generico}
\end{algorithm}

\subsection{Faster-All-Pairs-Shortest-Path}\label{sec:faster}
O cálculo de centralidades é feito utilizando o algoritmo Faster-All-Pairs-Shortest-Path. O Algoritmo Faster-All-Pairs-Shortest-Path
possui ordem de complexidade igual a $\theta(n^3log_2n)$. No melhor caso apenas $k$ arestas serão retiradas do grafo e no pior
caso $m$ arestas serão retiradas do grafo. Assim o algoritmo de cálculo de comunidades utilizando o algoritmo Faster-All-Pairs-Shortest-Path terá ordem de complexidade igual a $\theta(kn^3log_2n)$ no melhor caso e $\theta(mn^3log_2n)$ no pior caso. A quantidade
de memória utilizada por esse algoritmo será da ordem de $\theta(n^2)$, que é a ordem de utilização de memória dado pela estrutura
de dados grafo.

\subsection{Floyd-Warshall}\label{sec:floydwarshall}
O cálculo de centralidades é feito utilizando o algoritmo de Floydwarshall. O Algoritmo Floydwarshall
possui ordem de complexidade igual a $\theta(n^3)$. No melhor caso apenas $k$ arestas serão retiradas do grafo e no pior
caso $m$ arestas serão retiradas do grafo. Assim o algoritmo de cálculo de comunidades utilizando o algoritmo Floydwarshall terá ordem de complexidade igual a $\theta(kn^3)$ no melhor caso e $\theta(mn^3)$ no pior caso. A quantidade
de memória utilizada por esse algoritmo será da ordem de $\theta(n^2)$, que é a ordem de utilização de memória dado pela estrutura
de dados grafo.

\subsection{Johnson com Fila de Prioridades Mínimas}\label{sec:johnson_queue}
O cálculo de centralidades é feito utilizando o algoritmo de Johnson que utiliza o algoritmo de Dijkstra com fila
de prioridade mínima. Nesse caso o Algoritmo de Johnson que utiliza o algoritmo de Dijkstra com fila de prioridades mínima
possui ordem de complexidade igual a $O(nmlog_2n)$. No melhor caso apenas $k$ arestas serão retiradas do grafo e no pior
caso $m$ arestas serão retiradas do grafo. Assim o algoritmo de cálculo de comunidades utilizando o algoritmo de Johnson terá ordem de complexidade igual a $O(knmlog_2n)$ no melhor caso e $O(nm^2log_2n)$ no pior caso. A quantidade
de memória utilizada por esse algoritmo será da ordem de $\theta(n^2)$, que é a ordem de utilização de memória dado pela estrutura
de dados grafo.

\subsection{Johnson com Array}\label{sec:johnson_vector}
O cálculo de centralidades é feito utilizando o algoritmo de Johnson que utiliza o algoritmo de Dijkstra com um array. Nesse caso o Algoritmo de Johnson possui ordem de complexidade igual a $\O(n^3 + nm)$. No melhor caso apenas $k$ arestas serão retiradas do grafo e no pior caso $m$ arestas serão retiradas do grafo. Assim o algoritmo de cálculo de comunidades utilizando o algoritmo de Johnson terá ordem de complexidade igual a $O(k(n^3 + nm))$ no melhor caso e $O(mn^3 + nm^2)$ no pior caso. A quantidade
de memória utilizada por esse algoritmo será da ordem de $\theta(n^2)$, que é a ordem de utilização de memória dado pela estrutura
de dados grafo.

\subsection{Breadth First Search}\label{sec:bfs}
O cálculo de centralidades é feito utilizando o algoritmo Breadth First Search (BFS). O Algoritmo BFS
possui ordem de complexidade igual a $O(n + m)$. No cálculo de centralidade utilizando o algoritmo BFS
para cada vértice $v \in V$ é executado o algoritmo BFS. No melhor caso apenas $k$ arestas serão retiradas do grafo e no pior
caso $m$ arestas serão retiradas do grafo. Assim o algoritmo de cálculo de comunidades utilizando o algoritmo BFS terá ordem de complexidade igual a $O(k(n^2 + nm))$ no melhor caso e $O(mn^2 + nm^2))$ no pior caso. A quantidade
de memória utilizada por esse algoritmo será da ordem de $\theta(n^2)$, que é a ordem de utilização de memória dado pela estrutura
de dados grafo.
\section{Experimentos Computacionais}\label{sec:experimentos} 
Os experimentos computacionais foram executados em uma máquina Intel Dual-Core de 2.81 GHz de clock e
2GB de memória RAM, rodando o sistema operacional Linux. O modelo matemático apresentado em \ref{sec:modelagem} foi implementado
no Ilog CPLEX 12.5.1 e o algoritmo da seção \ref{sec:algoritmo} foi implementado
em Python 2.7, sendo que o otimizador utilizado para resolver os modelos lineares presente nesse algoritmo foi o Ilog CPLEX 12.5.1.
Foram utilizados quatro conjuntos de instâncias de testes nos experimentos computacionais e essas instâncias foram retiradas de \cite{Beasley90}.
Nessas instâncias de testes cada linha da matriz de incidência $A$ é coberta por pelo menos duas colunas e cada coluna cobre pelo menos uma linha. O
custo $c_j$ de cada coluna $j$ está entre $[1,100]$. A tabela \ref{table:instancias} resume esses conjuntos de instâncias. A coluna 1 dessa tabela 
representa o identificador do conjunto da instância de teste,
as colunas 2 e 3 mostram, respectivamente, o número $m$ de linhas e $n$ de colunas da matriz de incidência $A$. A coluna 4 representa
a densidade da matriz $A$ que é calculado pelo quantidade de 1's dessa matriz dividido pela quantidade total de elementos de $A$ que é igual a 
$mn$ e a coluna 5 mostra a quantidade de problemas em cada conjunto. Os dez problemas do conjunto 4 são nomeados como scp41-scp410, os cinco
problemas do conjunto 6 são nomeados como scp61-scp65, e os problemas do conjunto A e B são nomeados respectivamente como scpa1-scpa5
e scpb1-scpb5.

\begin{table}[htbp]
\begin{center}
  \begin{tabular}{|c|r|r|r|r|}
    \hline
      Conjunto & Linhas   & Colunas & Densidade   & Problemas\\ \hline
      4        & 200      & 1000    & 2           & 10 \\ \hline
      6        & 200      & 1000    & 5           & 5 \\ \hline
      A        & 300      & 3000    & 2           & 5 \\ \hline
      B        & 300      & 3000    & 5           & 5 \\ \hline
  \end{tabular}
\caption{Detalhes das instâncias de testes utilizadas}
\label{table:instancias}
\end{center}
\end{table}
No experimento desse trabalho foi comparado a performance do modelo proposto na seção \ref{sec:modelagem}, que será 
chamado aqui de $IP$, com o algoritmo proposto na seção \ref{sec:algoritmo}, que será chamado $ARank1$.
O modelo $IP$ foi executado através do CPLEX com todos os parâmetros default. O modelo presente no algoritmo $ARank1$
foi executado pelo CPLEX com um tempo máximo de 120 segundos e foi setado para que o CPLEX encontra-se no máximo cinco soluções
inteiras, explorando no máximo 50000 nós na árvore de Branch-and-Bound, encontra-se soluções com um valor objetivo sempre inferior a $-0.05$
e a ênfase na busca de soluções foi setado 4. O modelo $IP$ e o algoritmo $ARank1$ foram executados com um tempo de execução máximo de 7200 segundos. \\
A tabela \ref{table:resultados4e6} apresenta os resultados obtidos para o conjuntos de instância 4 e 6 e a tabela
\ref{table:resultadosaeb} apresenta os resultados para o conjuntos de instância A e B. Nessas tabelas a
coluna 1 mostra o nome da instância de teste, as colunas 2 e 3 são resultados referentes ao modelo $IP$ e as colunas
4,5,6 e 7 são resultados referentes ao algoritmo $ARank1$. A coluna 2 apresenta o custo da solução obtido pela
modelo $IP$ e a coluna 3 apresenta o tempo consumido para encontrar essa solução. A coluna 4 mostra o valor da
relaxação linear obtida para o $SCP$ no início do algoritmo $ARank1$, a coluna 5 apresenta o custo da solução 
obtido após procurar e inserir os cortes de Chvátal-Gomory de rank 1, a coluna 6 traz a quantidade de cortes que o algoritmo
$ARank1$ conseguiu adicionar e a coluna 7 mostra o tempo consumido pelo algoritmo $ARank1$.\\
Para todas as instâncias do conjunto 4 e 6 o CPLEX conseguiu encontrar soluções ótimas em um tempo muito pequeno, conforme
pode ser observado pelas colunas 2 e 3 da tabela \ref{table:resultados4e6}. Para as instâncias scp41, scp42, scp43, scp45
e scp47 nenhum corte é possível de ser adicionado, pois a relaxação linear já provém uma solução inteira ótima para o $SCP$, conforme
pode ser observado na coluna 4, linhas 1,2,3,5 e 7 da tabela \ref{table:resultados4e6}. Para todas as outras instâncias
desse conjunto o algoritmo $ARank1$ conseguiu adicionar cortes de Chvátal-Gomory de rank-1, como pode ser observado
pelas linhas 4,6,8,9,10,11,12,13,14 e 15 na coluna 6 da tabela \ref{table:resultados4e6}. Para esse conjunto, a única 
instância que o algoritmo $ARank1$ conseguiu resolver na otimalidade foi a instância scp410, conforme pode ser observado
pela coluna 5, linha 10 da tabela \ref{table:resultados4e6}. Nessa mesma tabela, retirando-se as instâncias que possuem
uma relaxação linear inteira,  pode-se observar que o tempo gasto pelo algoritmo $ARank1$ foi bastane alto, conforme a coluna
7.

\begin{table}[htbp]
\begin{center}
  \begin{tabular}{|c|r|r|r|r|r|r|}
    \hline
      Instância & \multicolumn{2}{|c|}{$IP$} & \multicolumn{4}{|c|}{$ARank1$}\\
                & Custo Solução    & Tempo(s)   & Relaxação Linear  & Custo Solução   & \#Cortes & Tempo(s)      \\ \hline
      scp41     & 429              & 0.84      & 429.00            & 429.00          & 0        & 0.00       \\ \hline
      scp42     & 512              & 0.85      & 512.00            & 512.00          & 0        & 0.00       \\ \hline
      scp43     & 516              & 0.86      & 516.00            & 516.00          & 0        & 0.00       \\ \hline
      scp44     & 494              & 0.86      & 494.00            & 494.00          & 6        & 863.71     \\ \hline
      scp45     & 512              & 0.85      & 512.00            & 512.00          & 0        & 0.00       \\ \hline
      scp46     & 560              & 0.89      & 557.25            & 558.94          & 32       & 1038.07    \\ \hline
      scp47     & 430              & 0.83      & 430.00            & 430.00          & 0        & 0.00       \\ \hline
      scp48     & 492              & 0.97      & 488.67            & 490.67          & 65       & 6118.55    \\ \hline
      scp49     & 641              & 0.90      & 638.54            & 639.87          & 75       & 8054.13    \\ \hline
      scp410    & 514              & 0.90      & 513.50            & 514.00          & 14       & 1349.04    \\ \hline
      scp61     & 138              & 1.23      & 133.14            & 133.53          & 52       & 7267.44    \\ \hline
      scp62     & 146              & 2.06      & 140.46            & 141.15          & 50       & 7284.06    \\ \hline
      scp63     & 145              & 1.26      & 140.13            & 141.46          & 72       & 7289.33    \\ \hline
      scp64     & 131              & 0.96      & 129.00            & 130.08          & 77       & 7251.63    \\ \hline
      scp65     & 161              & 1.94      & 153.35            & 154.13          & 50       & 7370.46    \\ \hline
  \end{tabular}
\caption{Comparação entre os custos da solução e tempos obtidos entre o modelo $IP$ e o algoritmo $ARank1$ para as instâncias do conjunto 4 e 6.}
\label{table:resultados4e6}
\end{center}
\end{table}

Para todas as instâncias do conjunto A e B o CPLEX conseguiu encontrar soluções ótimas em um tempo pequeno, conforme
pode ser observado pelas colunas 2 e 3 da tabela \ref{table:resultadosaeb}. Para essas instâncias o algoritmo {$ARank1$}
não conseguiu encontrar a solução ótima para nenhuma delas, conforme pode ser observado pela coluna 5 da 
tabela \ref{table:resultadosaeb}. O algoritmo {$ARank1$} só conseguiu encontrar cortes de Chvátal-Gomory de rank-1 
para as instâncias scpa3, scpa5 e scpb5, conforme pode ser observado pela coluna 7 e linhas 3,5 e 10 da 
tabela \ref{table:resultadosaeb}. O tempo máximo de 120 segundos para o modelo de separação no algoritmo {$ARank1$}
pode ter sido a causa de não encontrar cortes válidos para as demais instâncias. Para a instância scpa3, mesmo
adicionando cortes válidos a solução encontrada possui o mesmo custo da solução na relaxação linear, conforme pode 
ser observado, comparando-se a coluna 4 e 5 da linha 3, indicando degenerações nas soluções.

\begin{table}[htbp]
\begin{center}
  \begin{tabular}{|c|r|r|r|r|r|r|}
    \hline
      Instância & \multicolumn{2}{|c|}{$IP$} & \multicolumn{4}{|c|}{$ARank1$}\\
                & Custo Solução    & Tempo(s)   & Relaxação Linear  & Custo Solução   & \#Cortes & Tempo(s)      \\ \hline
      scpa1     & 253              & 9.95      & 246.84          & 246.84          & 0          & 251.75  \\ \hline
      scpa2     & 252              & 9.87      & 247.50          & 247.50          & 0          & 252.95  \\ \hline
      scpa3     & 232              & 9.48      & 228.00          & 228.00          & 6          & 1685.40 \\ \hline
      scpa4     & 234              & 8.76      & 231.40          & 231.40          & 0          & 248.60  \\ \hline
      scpa5     & 236              & 8.64      & 234.89          & 235.02          & 25         & 5826.96 \\ \hline
      scpb1     & 69               & 10.08     & 64.54           & 64.54           & 0          & 246.67  \\ \hline
      scpb2     & 76               & 10.87     & 69.30           & 69.30           & 0          & 256.28  \\ \hline
      scpb3     & 80               & 9.67      & 74.16           & 74.16           & 0          & 252.03  \\ \hline
      scpb4     & 79               & 11.52     & 71.22           & 71.22           & 0          & 250.54  \\ \hline
      scpb5     & 72               & 9.86      & 67.67           & 67.67           & 2          & 741.67  \\ \hline
  \end{tabular}
\caption{Comparação entre os custos da solução e tempos obtidos entre o modelo $IP$ e o algoritmo $ARank1$ para as instâncias do conjunto A e B.}
\label{table:resultadosaeb}
\end{center}
\end{table}














\bibliographystyle{plain}
%\chapbblname{tp}  % The name of the .bbl file, what is normally the name of your .tex file.
\bibliography{Tp} %\chapbibliography{gow}

\end{document}





% LocalWords:  comeca ij kj ji lll hardcoded eq ranqueamento recuperável RRSP
% LocalWords:  Recuperável Rent Minmax Regret Single Pair MSP SPP Cplex
