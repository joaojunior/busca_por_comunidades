\section{Experimentos Computacionais}\label{sec:experimentos} 
Os experimentos computacionais foram executados em uma máquina Intel Dual-Core de 2.81 GHz de clock e
2GB de memória RAM, rodando o sistema operacional Linux. O modelo matemático apresentado em \ref{sec:modelagem} foi implementado
no Ilog CPLEX 12.5.1 e o algoritmo da seção \ref{sec:algoritmo} foi implementado
em Python 2.7, sendo que o otimizador utilizado para resolver os modelos lineares presente nesse algoritmo foi o Ilog CPLEX 12.5.1.
Foram utilizados quatro conjuntos de instâncias de testes nos experimentos computacionais e essas instâncias foram retiradas de \cite{Beasley90}.
Nessas instâncias de testes cada linha da matriz de incidência $A$ é coberta por pelo menos duas colunas e cada coluna cobre pelo menos uma linha. O
custo $c_j$ de cada coluna $j$ está entre $[1,100]$. A tabela \ref{table:instancias} resume esses conjuntos de instâncias. A coluna 1 dessa tabela 
representa o identificador do conjunto da instância de teste,
as colunas 2 e 3 mostram, respectivamente, o número $m$ de linhas e $n$ de colunas da matriz de incidência $A$. A coluna 4 representa
a densidade da matriz $A$ que é calculado pelo quantidade de 1's dessa matriz dividido pela quantidade total de elementos de $A$ que é igual a 
$mn$ e a coluna 5 mostra a quantidade de problemas em cada conjunto. Os dez problemas do conjunto 4 são nomeados como scp41-scp410, os cinco
problemas do conjunto 6 são nomeados como scp61-scp65, e os problemas do conjunto A e B são nomeados respectivamente como scpa1-scpa5
e scpb1-scpb5.

\begin{table}[htbp]
\begin{center}
  \begin{tabular}{|c|r|r|r|r|}
    \hline
      Conjunto & Linhas   & Colunas & Densidade   & Problemas\\ \hline
      4        & 200      & 1000    & 2           & 10 \\ \hline
      6        & 200      & 1000    & 5           & 5 \\ \hline
      A        & 300      & 3000    & 2           & 5 \\ \hline
      B        & 300      & 3000    & 5           & 5 \\ \hline
  \end{tabular}
\caption{Detalhes das instâncias de testes utilizadas}
\label{table:instancias}
\end{center}
\end{table}
No experimento desse trabalho foi comparado a performance do modelo proposto na seção \ref{sec:modelagem}, que será 
chamado aqui de $IP$, com o algoritmo proposto na seção \ref{sec:algoritmo}, que será chamado $ARank1$.
O modelo $IP$ foi executado através do CPLEX com todos os parâmetros default. O modelo presente no algoritmo $ARank1$
foi executado pelo CPLEX com um tempo máximo de 120 segundos e foi setado para que o CPLEX encontra-se no máximo cinco soluções
inteiras, explorando no máximo 50000 nós na árvore de Branch-and-Bound, encontra-se soluções com um valor objetivo sempre inferior a $-0.05$
e a ênfase na busca de soluções foi setado 4. O modelo $IP$ e o algoritmo $ARank1$ foram executados com um tempo de execução máximo de 7200 segundos. \\
A tabela \ref{table:resultados4e6} apresenta os resultados obtidos para o conjuntos de instância 4 e 6 e a tabela
\ref{table:resultadosaeb} apresenta os resultados para o conjuntos de instância A e B. Nessas tabelas a
coluna 1 mostra o nome da instância de teste, as colunas 2 e 3 são resultados referentes ao modelo $IP$ e as colunas
4,5,6 e 7 são resultados referentes ao algoritmo $ARank1$. A coluna 2 apresenta o custo da solução obtido pela
modelo $IP$ e a coluna 3 apresenta o tempo consumido para encontrar essa solução. A coluna 4 mostra o valor da
relaxação linear obtida para o $SCP$ no início do algoritmo $ARank1$, a coluna 5 apresenta o custo da solução 
obtido após procurar e inserir os cortes de Chvátal-Gomory de rank 1, a coluna 6 traz a quantidade de cortes que o algoritmo
$ARank1$ conseguiu adicionar e a coluna 7 mostra o tempo consumido pelo algoritmo $ARank1$.\\
Para todas as instâncias do conjunto 4 e 6 o CPLEX conseguiu encontrar soluções ótimas em um tempo muito pequeno, conforme
pode ser observado pelas colunas 2 e 3 da tabela \ref{table:resultados4e6}. Para as instâncias scp41, scp42, scp43, scp45
e scp47 nenhum corte é possível de ser adicionado, pois a relaxação linear já provém uma solução inteira ótima para o $SCP$, conforme
pode ser observado na coluna 4, linhas 1,2,3,5 e 7 da tabela \ref{table:resultados4e6}. Para todas as outras instâncias
desse conjunto o algoritmo $ARank1$ conseguiu adicionar cortes de Chvátal-Gomory de rank-1, como pode ser observado
pelas linhas 4,6,8,9,10,11,12,13,14 e 15 na coluna 6 da tabela \ref{table:resultados4e6}. Para esse conjunto, a única 
instância que o algoritmo $ARank1$ conseguiu resolver na otimalidade foi a instância scp410, conforme pode ser observado
pela coluna 5, linha 10 da tabela \ref{table:resultados4e6}. Nessa mesma tabela, retirando-se as instâncias que possuem
uma relaxação linear inteira,  pode-se observar que o tempo gasto pelo algoritmo $ARank1$ foi bastane alto, conforme a coluna
7.

\begin{table}[htbp]
\begin{center}
  \begin{tabular}{|c|r|r|r|r|r|r|}
    \hline
      Instância & \multicolumn{2}{|c|}{$IP$} & \multicolumn{4}{|c|}{$ARank1$}\\
                & Custo Solução    & Tempo(s)   & Relaxação Linear  & Custo Solução   & \#Cortes & Tempo(s)      \\ \hline
      scp41     & 429              & 0.84      & 429.00            & 429.00          & 0        & 0.00       \\ \hline
      scp42     & 512              & 0.85      & 512.00            & 512.00          & 0        & 0.00       \\ \hline
      scp43     & 516              & 0.86      & 516.00            & 516.00          & 0        & 0.00       \\ \hline
      scp44     & 494              & 0.86      & 494.00            & 494.00          & 6        & 863.71     \\ \hline
      scp45     & 512              & 0.85      & 512.00            & 512.00          & 0        & 0.00       \\ \hline
      scp46     & 560              & 0.89      & 557.25            & 558.94          & 32       & 1038.07    \\ \hline
      scp47     & 430              & 0.83      & 430.00            & 430.00          & 0        & 0.00       \\ \hline
      scp48     & 492              & 0.97      & 488.67            & 490.67          & 65       & 6118.55    \\ \hline
      scp49     & 641              & 0.90      & 638.54            & 639.87          & 75       & 8054.13    \\ \hline
      scp410    & 514              & 0.90      & 513.50            & 514.00          & 14       & 1349.04    \\ \hline
      scp61     & 138              & 1.23      & 133.14            & 133.53          & 52       & 7267.44    \\ \hline
      scp62     & 146              & 2.06      & 140.46            & 141.15          & 50       & 7284.06    \\ \hline
      scp63     & 145              & 1.26      & 140.13            & 141.46          & 72       & 7289.33    \\ \hline
      scp64     & 131              & 0.96      & 129.00            & 130.08          & 77       & 7251.63    \\ \hline
      scp65     & 161              & 1.94      & 153.35            & 154.13          & 50       & 7370.46    \\ \hline
  \end{tabular}
\caption{Comparação entre os custos da solução e tempos obtidos entre o modelo $IP$ e o algoritmo $ARank1$ para as instâncias do conjunto 4 e 6.}
\label{table:resultados4e6}
\end{center}
\end{table}

Para todas as instâncias do conjunto A e B o CPLEX conseguiu encontrar soluções ótimas em um tempo pequeno, conforme
pode ser observado pelas colunas 2 e 3 da tabela \ref{table:resultadosaeb}. Para essas instâncias o algoritmo {$ARank1$}
não conseguiu encontrar a solução ótima para nenhuma delas, conforme pode ser observado pela coluna 5 da 
tabela \ref{table:resultadosaeb}. O algoritmo {$ARank1$} só conseguiu encontrar cortes de Chvátal-Gomory de rank-1 
para as instâncias scpa3, scpa5 e scpb5, conforme pode ser observado pela coluna 7 e linhas 3,5 e 10 da 
tabela \ref{table:resultadosaeb}. O tempo máximo de 120 segundos para o modelo de separação no algoritmo {$ARank1$}
pode ter sido a causa de não encontrar cortes válidos para as demais instâncias. Para a instância scpa3, mesmo
adicionando cortes válidos a solução encontrada possui o mesmo custo da solução na relaxação linear, conforme pode 
ser observado, comparando-se a coluna 4 e 5 da linha 3, indicando degenerações nas soluções.

\begin{table}[htbp]
\begin{center}
  \begin{tabular}{|c|r|r|r|r|r|r|}
    \hline
      Instância & \multicolumn{2}{|c|}{$IP$} & \multicolumn{4}{|c|}{$ARank1$}\\
                & Custo Solução    & Tempo(s)   & Relaxação Linear  & Custo Solução   & \#Cortes & Tempo(s)      \\ \hline
      scpa1     & 253              & 9.95      & 246.84          & 246.84          & 0          & 251.75  \\ \hline
      scpa2     & 252              & 9.87      & 247.50          & 247.50          & 0          & 252.95  \\ \hline
      scpa3     & 232              & 9.48      & 228.00          & 228.00          & 6          & 1685.40 \\ \hline
      scpa4     & 234              & 8.76      & 231.40          & 231.40          & 0          & 248.60  \\ \hline
      scpa5     & 236              & 8.64      & 234.89          & 235.02          & 25         & 5826.96 \\ \hline
      scpb1     & 69               & 10.08     & 64.54           & 64.54           & 0          & 246.67  \\ \hline
      scpb2     & 76               & 10.87     & 69.30           & 69.30           & 0          & 256.28  \\ \hline
      scpb3     & 80               & 9.67      & 74.16           & 74.16           & 0          & 252.03  \\ \hline
      scpb4     & 79               & 11.52     & 71.22           & 71.22           & 0          & 250.54  \\ \hline
      scpb5     & 72               & 9.86      & 67.67           & 67.67           & 2          & 741.67  \\ \hline
  \end{tabular}
\caption{Comparação entre os custos da solução e tempos obtidos entre o modelo $IP$ e o algoritmo $ARank1$ para as instâncias do conjunto A e B.}
\label{table:resultadosaeb}
\end{center}
\end{table}












